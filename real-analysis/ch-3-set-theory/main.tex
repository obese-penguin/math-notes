\documentclass[12pt]{article}
\usepackage{graphicx} % Required for inserting images
\usepackage[a4paper, total={6in, 10in}]{geometry}
\usepackage{datetime, xcolor, tcolorbox}
\usepackage{amsthm, thmtools, amsfonts, mathtools, amssymb}
\usepackage[shortlabels]{enumitem}

\tcbuselibrary{theorems}

\newdateformat{monthyeardate}{\monthname[\THEMONTH] \THEYEAR}

\newtheorem{theorem}{Theorem}[section]
\newtheorem{corollary}{Corollary}[theorem]
\newtheorem{lemma}[theorem]{Lemma}
\newtheorem{axiom}{Axiom}
\newtheorem{proposition}{Proposition}[section]
%\newtheorem{definition}{Definition}[section]
\theoremstyle{remark}
\newtheorem*{remark}{Remark}
\newtheorem*{notation}{Notation}

\definecolor{box-head}{rgb}{0, 0.435, 0.510}
\definecolor{box-body}{rgb}{0.506, 0.718, 0.757}

\newtcolorbox{important}[1][]{colframe=box-head, colback=box-body!28!, title=#1, fonttitle=\bfseries, before skip = 10pt}

\declaretheoremstyle[
    spaceabove=5pt, 
    spacebelow=6pt, 
    headfont=\normalfont\bfseries,
    notefont=\normalfont\bfseries,
    bodyfont = \normalfont,
    headpunct={:}]{definitionstyle} 
\declaretheorem[name={Definition}, style=definitionstyle, numberwithin=section]{definition}
\declaretheorem[name={Example}, style=definitionstyle, numberwithin=section]{example}

\title{Set Theory}
\author{Riddhiman}
\date{\monthyeardate\today}

\begin{document}

\def\inc{{\mathrel+\joinrel\mathrel+}}  

\maketitle

\section{Preface}%
\label{sec:Preface}

This is approximately a $ 40 $ page long chapter on introductory set theory. Brace yourselves this might get very boring. Here we basically take a break from developing different number systems to define what a set actually is. Note that we didn't really use the set notation or terminology in the previous chapter, except in some informal examples. Set theory will make it much more convenient for us to talk about number systems in the future and also to develop the notion of functions and mappings.  

\section{Fundamentals}%
\label{sec:Fundamentals}
 
\begin{definition}[Informal]
	We define a \textit{set} $ A $ to be any unordered collection of ``objects''. If $ x $ is an object, we say that $ x $ \textit{is an element of} $ A $ and denote it as $ x \in A $. Otherwise we denote it as $ x \notin A $.
\end{definition}
 
This is informal since the distinction between sets and objects has not been made among other things. We take an axiomatic approach to resolve this. 

% axiom 1

\begin{important}[Axiom 1: Sets are objects]
     If $ A $ is a set, then $ A $ is also an object. In particular, given two sets $ A $ and $ B $, it is meaningful to ask whether $ A $ is also an element of $ B $. 
\end{important}

\begin{definition}[Equality of sets]
    Two sets $ A $ and $ B $ are said to be equal iff every element of $ A $ is an element of $ B $ and vice versa.
\end{definition}

\begin{notation}
    Note that we will denote the empty set as $ \emptyset $.
\end{notation}

% axiom 2

\begin{important}[Axiom 2: Empty Set]
     There exists a set $ \emptyset $, known as the empty set, which contains no elements. 
\end{important}


This next axiom is a bit weird. I'm not sure what it was trying to achieve and felt a little redundant. It felt like it could have had a bigger reach but instead left room for more axioms. The author did say that some axioms will be redundant. In the book it's quite long and verbose for an axiom. I provide the shortened version here, but it conveys the same information. 

% axiom 3

\begin{important}[Axiom 3: Singleton set and pair sets]
 	If $ a $ is an object, then there exists a set $ \{a\} $ whose only element is $ a $; we refer to $ \{a\} $ as the singleton set whose element is $ a $. Furthermore, if $ a $ and $ b $ are objects, then there exists a set $ \{a,b\} $ whose only elements are $ a $ and $ b $; we refer to this as the pair set formed by $ a $ and $ b $.
\end{important}

As you can see it also introduces terminology that isn't really commonly used.(?) \par 

Now we build slightly larger sets. Note that ``cardinality'' of a set has not been defined yet. 

% axiom 4

\begin{important}[Axiom 4: Pairwise union]
 	Given any two sets $ A $, $ B $, there exists a set $ A \cup B $ called the union of $ A $ and $ B $. It is defined as \[ x \in A \cup B \iff (x \in A \textit{ or } x \in B). \]
\end{important}


\begin{proposition}[Union operation is commutative and associative]
    If we have sets $ A $, $ B $ and $ C $, then $ A \cup B = B \cup A $ and $ (A \cup B) \cup C = A \cup (B \cup C) $.
\end{proposition}

This can be proven with Definition 2.2 and Axiom 3.


\begin{definition}[Subsets]
	Let $ A $, $ B $ be sets. We say that $ A $ is a \textit{subset} of $ B $, denoted by $ A \subseteq B $, iff \[ \text{For any object } x, x \in A \implies x \in B. \]
	We say that $ A $ is a proper subset of $ B $, denoted by $ A \subsetneq B $ if $ A \subseteq B $ and $ A \neq B $.
\end{definition}

% put note about subset notation here

\begin{proposition}[Sets are partially ordered by set inclusion]
    Let $ A $, $ B $, $ C $ be sets. Then if $ A \subseteq B $ and $ B \subseteq C $ then $ A \subseteq C $. If $ A \subseteq B $ and $ B \subseteq A $ then $ A = B $. Also if $ A \subsetneq B $ and $ B \subsetneq C $ then $ A \subsetneq C $.
\end{proposition}

This can be proven easily by the Definition 2.3.

\begin{important}[Axiom 5: Axiom of specification]
	Let $ A $ be a set and for every $ x \in A $ let there be a property $ P(x) $ pertaining to $ x $. Then there exists a set, called $ \{ x \in A : P(x) \text{ is true} \} $, whose elements are precisely the elements of $ A $ for which $ P(x) $ is true.  
\end{important}

Alternate phrasing of the same: To every set $ A $ and to every property $ P(x) $ there exists a set $ B $ whose elements are exactly those elements of $ A $ for which $ P(x) $ holds. 

\begin{definition}[Intersections]
	The intersection $ S_{1} \cap S_{2} $ of two sets is defined to be the set 
	\begin{equation*}
		S_{1} \cap S_{2} \coloneqq \{ x \in S_{1} : x \in S_{2} \}.
	\end{equation*}
\end{definition}

\begin{definition}[Difference sets]
    Given two sets $ A $ and $ B $, we define the set $ A - B $ or $ A \ B $ to be the set $ A $ with any elements of $ B $ removed: 
    \begin{equation*}
        A - B \coloneqq {x \in A : x \notin B}.
    \end{equation*}
\end{definition}

\begin{proposition}[Sets form a boolean algebra]
    Let $ A $, $ B $, $ C $ be sets, and let $ X $ be a set containing $ A $, $ B $, $ C $ as subsets. 
    \begin{enumerate}[(a)]
	\item (Minimal Element) $ A \cup \emptyset = A $ and $ A \cap \emptyset = \emptyset $.
	\item (Maximal element) $ A \cup X = X $ and $ A \cap X = A $.
	\item (Identity) $ A \cup A = A $ and $ A \cap A = A $.
	\item (Commutativity) $ A \cup B = B \cup A $ and $ A \cap B = B \cap A $.
	\item (Associativity) Self explanatory. C'mon I'm not gonna write everything.
	\item (Distributivity) $ A \cap (B \cup C) = (A \cap B) \cup (A \cap C) $ and $ A \cup (B \cap C) = (A \cup B) \cap (A \cup C) $.
	\item (Partition) $ A \cup (X - A) = X $ and $ A \cap (X - A) = \emptyset $.
	\item (De Morgan laws) $ X - (A \cup B) = (X - A) \cap (X - B) $ and $ X - (A \cap B) = (X - A) \cup (X - B) $.
    \end{enumerate}
\end{proposition}

This next axiom feels a bit like its creating some notion of function, domain and range. 

\begin{important}[Axiom 6: Replacement]
	Let $ A $ be a set. For any object $ x \in A $, and any object $ y $, we have a statement $ P(x, y) $ pertaining to $ x $ and $ y $, such that for each $ x \in A $ there is atmost one $ y $ for which $ P(x, y) $ is true. Then there exists a set $ \{ y : P(x, y) \text{ is true for some } x \in A \} $, such that for any object $ z $, 
	\begin{equation*}
		z \in \{y : P(x, y) \text{ is true for some } x \in A \} \iff P(x, z) \text{ is true for some } x \in A.
	\end{equation*}
	
\end{important}

Formalising the set of natural numbers next, 

\begin{important}[Axiom 7: Infinity]
	There exists a set $ \mathbb{N} $, whose elements are called natural numbers, as well as an object $ 0 $ in $ \mathbb{N} $, and an object $ n\inc $ assigned to every natural number $ n \in \mathbb{N} $, such that the Peano axioms hold.
\end{important}

\section{Russell's paradox}%
\label{sec:Russell's paradox}

Throughout we have been creating axioms that allow us to create larger and larger sets (maybe like axioms 3, 4, and 5).  The reason for such a controlled increase in the type and size of sets we could create will be apparent after this axiom.

\begin{important}[Axiom 8: Universal specification]
	Suppose for every object $ x $ we have a property $ P(x) $ pertaining to $ x $ (so that for every $ x $, $ P(x) $ is either a true or false statement). Then there exists a set $ \{ x : P(x) \text{ is true. } \} $ such that for every object $ y $, 
	\begin{equation*}
		y \in \{ x : P(x) \text{ is true} \} \iff P(y) \text{ is true. }
	\end{equation*}
\end{important}

This axiom implies almost all other axioms above but cannot be introduced into set theory because it create a logical contradiction known as \textit{Russell's paradox}:\par
Let $ P(x) $ be the statement 
\begin{equation*}
	P(x) \iff ``x \text{ is a set, and } x \notin x";
\end{equation*}
thus $ P(x) $ is true only when a set $ x $ does not contain itself. Now consider the following set, 
\begin{equation*}
	\Omega \coloneqq \{ x : P(x) \text{ is true.} \} = \{ x : x \text{ is a set and } x \notin x \}.
\end{equation*}
This is the set of all sets that do not contain themselves. Since, by Axiom 8, $ \Omega $ is also a set, it is fair to ask if $ \Omega \in \Omega $. Note that $ \Omega $ cannot contain itself. If we assume that it does, then $ \Omega $ becomes the object in $ \Omega $ that contains itself, violating $ P(x) $. However, if $ \Omega \notin \Omega $ then again, by $ P(x) $, $ \Omega $ must contain itself. This is the paradox.  \par

The problem with the above axiom is that it can create sets far too large, sets that can include all kinds of objects. One way to informally resolve this issue is to create a hierarchy of objects. At the bottom are the \textit{primitive objects} -- the objects that are not sets, e.g. natural numbers. The next rung of the hierarchy are sets that only contain primitive objects, e.g. $ \{ 2, 3, 4 \} $ or $ \emptyset $; let these be called ``primitive'' sets. The next layer of the hierarchy contains set that have primitive sets as their objects and so on. This ensures that it is not possible to create sets that contain themselves.  

Anyways to counter absurdities like Russell's paradox, another axiom is postulated. 

\begin{important}[Axiom 9: Regularity]
    If $ A $ is a non-empty set, then there is atleast one element $ x $ of $ A $ which is either not a set, or is disjoint from $ A $.
\end{important}

The point of this axiom is that it is asserting that at least one of the elements of $ A $ is so low on the hierarchy of objects that it does not contain any other elements of $ A $. For example in $ A = \{ \{3, 4\}, \{3, 4, \{3, 4\}\} \} $, the element $ \{3, 4\} \in A $ does not contain any other element in $ A $.

\section{Functions}%
\label{sec:Functions}

\begin{definition}[Functions]
   Let $ X $, $ Y $ be sets, and let $ P(x, y) $ be a property pertaining to an object $ x \in X $ and an object $ y \in Y $ such that \textit{for every} $ x \in X $, there is \textit{exactly one} $ y \in Y $ for which $ P(x, y) $ is true. Then we define the \textit{function} $ f : X \rightarrow Y $ defined by $ P $ on the \textit{domain} $ X $ and \textit{range} Y to be the \textit{object} which, given any input $ x \in X $, assigns an output $ f(x) \in Y $, defined to be the unique object $ f(x) $ for which $ P(x, f(x)) $ is true. Thus, for any $ x \in X $ and $ y \in Y $, 
   \begin{equation*}
	   y = f(x) \iff P(x, y) \text{ is true}.
   \end{equation*}
   
\end{definition}

\begin{example}
	The increment operation $ \inc $ we used earlier is a function. For all $ x \in \mathbb{N} $ it guarantees a unique $ y \in \mathbb{N} $ (by Axiom 4 in previous chapter) such that $ y = x\inc $.
\end{example}

\begin{definition}[Equality of functions]
    Two functions $ f : X \rightarrow Y $, $ g : X \rightarrow Y $ with the same domain and range are said to be equal, $ f = g $, if and only if $ f(x) = g(x) $ for all $ x \in X $.
\end{definition}

\begin{definition}[Composition]
    Let $ f : X \rightarrow Y $ and $ g : Y \rightarrow Z $ be two functions, such that the range of $ f $ is the domain of $ g $. We then define the \textit{composition} $ g \circ f : X \rightarrow Z $ of the two functions $ g $ and $ f $ to be the function defined explicitly by the formula 
    \begin{equation*}
	    (g \circ f)(x) \coloneqq g(f(x)).
    \end{equation*}
\end{definition}

\begin{lemma}[Composition is associative]
	Let $ f : Z \rightarrow W $, $ g : Y \rightarrow Z $, $ h : X \rightarrow Y $  be functions. Then $ f \circ (g \circ h) = (f \circ g) \circ h $.   
\end{lemma}
This can be easily proven by the definition above.

\begin{definition}[One-to-one functions]
    A function $ f $ is \textit{one-to-one} (or \textit{injective}) if different elements map to different elements: 
    \[ x \neq x' \implies f(x) \neq f(x'). \]
    Equivalently, a function is one-to-one if 
    \[ f(x) = f(x') \implies x = x'. \]
\end{definition}

\begin{example}[Informal]
	The function $ f : \mathbb{Z} \rightarrow \mathbb{Z} $ such that $ f(x) = x^2 $ is not injective. The function $ g : \mathbb{N} \rightarrow \mathbb{Z} $, $ g(x) = x^2 $ is injective. 
\end{example}

\begin{definition}[Onto functions]
    A function $ f : X \rightarrow Y $ is \textit{onto} (or \textit{surjective}) if \textit{for every} $ y  \in Y $ there exists an $ x \in X $ such that $ f(x) = y $.
\end{definition}

\begin{example}
	The increment function in Example 4.1 is not surjective as there does not exist any $ x \in \mathbb{N} $ such that $ x\inc = 0 $ (by Axiom 3 in previous chapter). However, the function $ g : \mathbb{N} \rightarrow \mathbb{N} \backslash \{ 0 \} $ is surjective. 
\end{example}

\begin{definition}[Bijective functions]
    Functions $ f : X \rightarrow Y $ which are both one-to-one and onto are also called \textit{bijective} or \textit{invertible}.
\end{definition}

\begin{example}
    The function $ g $ in Example 4.3 is a bijection. 
\end{example}

If a function $ f $ is bijective, then \textit{for every} $ y \in Y $, there is \textit{exactly one} $ x $ such that $ f(x) = y $.

\end{document}
