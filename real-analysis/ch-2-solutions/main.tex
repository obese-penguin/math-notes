\documentclass[12pt]{article}
\usepackage{graphicx} % Required for inserting images
\usepackage[a4paper, total={6in, 10in}]{geometry}
\usepackage{datetime, xcolor, babel, enumitem}
\usepackage{amsthm, thmtools, amsfonts, mathtools}

\newdateformat{monthyeardate}{\monthname[\THEMONTH] \THEYEAR}

\newtheorem{theorem}{Theorem}[section]
\newtheorem{corollary}{Corollary}[theorem]
\newtheorem{lemma}[theorem]{Lemma}
\newtheorem{axiom}{Axiom}
\newtheorem{proposition}{Proposition}[section]
\newtheorem{definition}{Definition}[section]
\theoremstyle{remark}
\newtheorem*{remark}{Remark}


\newcommand{\exercise}[1]{\noindent {\bf Exercise #1.}}


\def\inc{{\mathrel+\joinrel\mathrel+}}  

\title{Chapter II - Solutions}
\author{Riddhiman}
\date{\monthyeardate\today}

\begin{document}

\maketitle

\section{Preface}%
\label{sec:Preface}

These are solutions to exercises in \textit{Analysis I} by Terence Tao. Most of them have been written by me, some with inputs from discussions on Math StackExchange. 

\section{Addition}%
\label{sec:Addition}

\exercise{2.2.1} Prove Proposition 2.2.5 which states that addition is associative. (Hint: fix two of the variables and induct on the third.)

\begin{proof}[Solution]
	Let $ a, b \text{ and } c $ be natural numbers and fix $ b $ and $ c $. We will induct over $ a $. For the base case, let $ a = 0 $. Hence from the definition of addition,  
	\begin{equation*}
		(0+b)+c = b+c = 0 + (b+c)
	\end{equation*}
	
	Now let the hypothesis be true for some arbitrary non-zero $ a $. Thus assume that $ (a+b)+c = a+(b+c) $. To close the induction we need to prove associativity for $ a\inc $. Since, $ (a\inc) + b = (a+b)\inc $, we can write
	\begin{equation}
		((a\inc) + b) + c = ((a+b)\inc) = ((a+b)+c)\inc
	\end{equation}
	
	Also, 
	
	\begin{equation}
		(a\inc) + (b+c) = (a+(b+c))\inc
	\end{equation}
	
	Since we assumed that $ a+(b+c) = (a+b)+c $, we can conclude that $ (1) = (2) $ (by Axiom 4?). This closes the induction.
	
\end{proof}

\exercise{2.2.2} Prove Lemma 2.2.10 which states that for a positive number $ a $ there exists only one natural number $ b $ such that $ b\inc = a $ (Hint: use induction.).

\begin{proof}[Solution] (by contradiction).
    Assume that there exists a natural number $ c $ such that $ c \neq b $ and $ b\inc = c\inc = a $. However, this violates Axiom 4 which states that if $ m\inc = n\inc $ then $ m=n $. Thus our assumption is false. 
\end{proof}
    
\exercise{2.2.3} Prove Proposition 2.2.12. (Hint: you will need many of the
preceding propositions, corollaries, and lemmas.)

\begin{proof}[Solution] \hfill
	\begin{enumerate}[label=(\alph*)]
		\item We know that if $ a = b + m $, $ m \geq 0 $ then $ a \geq b $. Let $ a = a + 0 $, then $ m = 0 $ and thus $ a \geq a $.

		\item From the definition we can write $ a = b+m $, $ m \geq 0 $ and $ b = c+n $, $ n \geq 0 $. Thus $ a = c+m+n $ which, again from the definition, gives $ a \geq c $.

		\item $ a \geq b \text{ gives } a = b + m $, $ m \geq 0 \text{ but } b \geq a \text{ gives } b = a + n $, $ n \geq 0. \text{ Thus, } a = a + m + n $ which, from Proposition 2.2.6, gives $ m + n = 0 $. Therefore $ m = n = 0 $ and $ a = b .$ 

		\item $ a \geq b \implies a+c \geq b+c $ direction: we have $ a = b+m $, $ m \geq 0  \implies a+c = (b+c)+m \geq b+c $ by definition. \\
			Opposite dirction: $ a + c \geq b + c \implies a + c = b + c $ \textbf{or} $ a + c > b + c $. From the first case we get, $ a = b $. From the second case, write $ a + c = (b + c) + m $, $ m > 0 $. Cancelling $ c $ we get, $ a = b + m $ which by definition is strictly greater than $ 0 $. Combining the two cases we get the desired inequality.

		\item $ a < b \implies b = a+m, m > 0 $. Note that there exists a predecessor of $ m $ thus $ m = n\inc $ and hence $ b = a+n\inc = (a+n)\inc \geq a\inc $
			
		\item Forward direction: Let $ b = a + m $, $ m \geq 0 $. But since by definition, $ a \neq b $, we cannot invoke the cancellation law to conclude $ m = 0 $. Hence, $ m > 0 $. \\
		      Backward direction: By definition, $ b \geq a $ but $ d \neq 0 \implies b \neq a$. Hence, discarding the $ a = b $ case we get $ b > a $. 
	\end{enumerate}

\end{proof}

\exercise{2.2.5} 
\begin{proof}[Solution]
    
\end{proof}


\exercise{2.2.6}
\begin{proof}[Solution]
   We proof this by induction and induct over $ n $. For the base case, let  $ n = 0 $, here $ m $ is also forced to be $ 0 $. So we need to prove that $ P(0) $ is true.

\end{proof}


\end{document}

