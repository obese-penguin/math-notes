\documentclass[12pt]{article}
\usepackage{graphicx} % Required for inserting images
\usepackage[a4paper, total={6in, 10in}]{geometry}
\usepackage{datetime}
\usepackage{xcolor}
\usepackage{amsthm, thmtools}

\newtheorem{theorem}{Theorem}[section]
\newtheorem{corollary}{Corollary}[theorem]
\newtheorem{lemma}[theorem]{Lemma}
\newtheorem{axiom}{Axiom}
\newtheorem{proposition}{Proposition}[section]
\newtheorem{definition}{Definition}[section]
\theoremstyle{remark}
\newtheorem*{remark}{Remark}
\newtheorem*{notation}{Notation}

\declaretheoremstyle[
  headfont=\color{blue}\normalfont\bfseries,
]{colored-example}

\declaretheorem[
  style=colored-example,
  name=Example,
  numberwithin=section
  ]{example}


\declaretheoremstyle[
  headfont=\color{red}\normalfont\bfseries,
]{colored-counter}

\declaretheorem[
  style=colored-counter,
  name=Counter Example,
  numberwithin=section
  ]{cexample}


\title{Chapter II - Natural Numbers}
\author{Riddhiman}
\date{July 2023}

\begin{document}

\def\inc{{\mathrel+\joinrel\mathrel+}}  

\maketitle

\section{Preface}%
\label{sec:Preface}
These are my self study Introductory Real Analysis notes. The source material is \textit{Analysis I} by Terrence Tao. This chapter is concerned with the construction of the Natural Numbers with the Peano Axioms and thereafter defining addition and multiplication on them. For these notes, after each Axiom we will provide \textcolor{blue}{\textbf{Examples}} and \textcolor{red}{\textbf{Counter Examples}} which will demonstrate the purpose of the axioms built and motivate the need for more axioms.

\section{The Peano axioms}%
\label{sec:The Peano axioms}

\begin{axiom}
   0 is a natural number. 
\end{axiom}

\begin{remark}
    Note that this point may be contentious as natural numbers generally exclude $0$. One can replace \textit{zero} with \textit{one} and the behaviour of the axioms would not change at all. However, the purpose of $0$ becomes apparent with the definition of addition, as $0$ is the additive identity. 
\end{remark}

\begin{axiom}
	If $n$ is a natural number, the $n\inc$ is also a natural number.
\end{axiom}

\begin{notation}
	Here `$\inc$' represents increment. `$n\inc$' represents the successor of $n$. 
\end{notation}

\begin{definition}
   Define $1 = 0\inc$, $2 = 1\inc$, $3 = 2\inc$ and so on.  
\end{definition}


\begin{cexample}
	Let the set of natural numbers be $\{0, 1, 2, 3\}$ and define $3\inc = 0$. Note that this is a perfectly valid set and doesnt violate Axioms 1 and 2. To prevent this kind of looping, we introduce 
\end{cexample}

\begin{axiom}
    0 is not the successor of any natural number.
\end{axiom}

\begin{proposition}
	4 is not equal to 0.  
\end{proposition}

\begin{proof}
    By definition, $4 = 3\inc$. By Axiom 1 and 2, 3 is natural number and by Axiom 3, $3\inc \neq 0$, thus $4 \neq 0$.
\end{proof}

\begin{cexample}
	Set of natural numbers is $\{0, 1, 2, 3, 4\}$ and define $4\inc = 1$.
\end{cexample}

\begin{axiom}
	Different natural numbers must have different successors. Let $n$, $m$ be two different natural numbers, then $n\inc \neq m\inc$. Equivalently, if $n\inc = m\inc$, then we must have $n=m$.
\end{axiom}

\begin{proposition}
    6 is not equal to 2.
\end{proposition}

\begin{proof}
    We proof by contradiction. Let $6 = 5\inc = 2 = 1\inc $. By Axiom 4, $5 = 1 \Rightarrow 4\inc = 0\inc \Rightarrow 4 = 0 \Rightarrow 3\inc = 0$ which contradicts Axiom 3.
\end{proof}

\begin{axiom}[Principle of mathematical induction]
	Let $P$ be any property pertaining to a natural number $n$. Suppose that $P(0)$ is true, and suppose that whenever $P(n)$ is true, $P(n\inc)$ is also true. Then $P$ is true for all natural numbers $n$. 
\end{axiom}


\end{document}








