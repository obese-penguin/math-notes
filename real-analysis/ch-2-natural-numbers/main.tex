\documentclass[12pt]{article}
\usepackage{graphicx} % Required for inserting images
\usepackage[a4paper, total={6in, 10in}]{geometry}
\usepackage{datetime}
\usepackage{xcolor}
\usepackage{amsthm, thmtools, amsfonts, enumitem}
\usepackage{mathtools}

\newdateformat{monthyeardate}{\monthname[\THEMONTH] \THEYEAR}

\newtheorem{theorem}{Theorem}[section]
\newtheorem{corollary}{Corollary}[theorem]
\newtheorem{lemma}[theorem]{Lemma}
\newtheorem{axiom}{Axiom}
\newtheorem{proposition}{Proposition}[section]
\newtheorem{definition}{Definition}[section]
\theoremstyle{remark}
\newtheorem*{remark}{Remark}
\newtheorem*{notation}{Notation}

\declaretheoremstyle[
  headfont=\color{blue}\normalfont\bfseries,
]{colored-example}

\declaretheorem[
  style=colored-example,
  name=Example,
  numberwithin=section
  ]{example}


\declaretheoremstyle[
  headfont=\color{red}\normalfont\bfseries,
]{colored-counter}

\declaretheorem[
  style=colored-counter,
  name=Counter Example,
  numberwithin=section
  ]{cexample}


\title{Chapter II - Natural Numbers}
\author{Riddhiman}
\date{\monthyeardate\today}

\begin{document}

\def\inc{{\mathrel+\joinrel\mathrel+}}  

\maketitle

\section{Preface}%
\label{sec:Preface}
These are my self study Introductory Real Analysis notes. The source material is \textit{Analysis I} by Terrence Tao. This chapter is concerned with the construction of the natural numbers with the Peano axioms and thereafter defining addition and multiplication on them. For these notes, after each axiom we will provide \textcolor{blue}{\textbf{Examples}} and \textcolor{red}{\textbf{Counter Examples}} which will demonstrate the purpose of the axioms built and motivate the need for more axioms.

\section{The Peano axioms}%
\label{sec:The Peano axioms}

\begin{axiom}
   0 is a natural number. 
\end{axiom}

\begin{remark}
    Note that this point may be contentious as natural numbers generally exclude $0$. We can replace \textit{zero} with \textit{one} and the behaviour of the axioms would not change at all. However, the purpose of $0$ becomes apparent with the definition of addition, as $0$ is the additive identity. 
\end{remark}

\begin{axiom}
	If $n$ is a natural number, the $n\inc$ is also a natural number.
\end{axiom}

\begin{notation}
	Here `$\inc$' represents increment. `$n\inc$' represents the successor of $n$. 
\end{notation}

\begin{definition}
   Define $1 = 0\inc$, $2 = 1\inc$, $3 = 2\inc$ and so on.  
\end{definition}


\begin{cexample}
	Let the set of natural numbers be $\{0, 1, 2, 3\}$ and define $3\inc = 0$. Note that this is a perfectly valid set and doesnt violate Axioms 1 and 2. To prevent this kind of looping, we introduce 
\end{cexample}

\begin{axiom}
    0 is not the successor of any natural number.
\end{axiom}

\begin{example}
	\textit{$4$ is not equal to $0$.}		  
\end{example}

\begin{proof}
    By definition, $4 = 3\inc$. By Axiom 1 and 2, 3 is natural number and by Axiom 3, $3\inc \neq 0$, thus $4 \neq 0$.
\end{proof}

\begin{cexample}
	Set of natural numbers is $\{0, 1, 2, 3, 4\}$ and define $4\inc = 1$.
\end{cexample}

\begin{axiom}
	Different natural numbers must have different successors. Let $n$, $m$ be two different natural numbers, then $n\inc \neq m\inc$. Equivalently, if $n\inc = m\inc$, then we must have $n=m$.
\end{axiom}

\begin{example}
	\textit{$6$ is not equal to $2$.}
\end{example}

\begin{proof}
    We proof by contradiction. Let $6 = 5\inc = 2 = 1\inc $. By Axiom 4, $5 = 1 \Rightarrow 4\inc = 0\inc \Rightarrow 4 = 0 \Rightarrow 3\inc = 0$ which contradicts Axiom 3.
\end{proof}

\begin{axiom}[Principle of mathematical induction]
	Let $P$ be any property pertaining to a natural number $n$. Suppose that $P(0)$ is true, and suppose that whenever $P(n)$ is true, $P(n\inc)$ is also true. Then $P$ is true for all natural numbers $n$. 
\end{axiom}

\begin{proposition}
    A certain property $ P(n) $ is true for every natural number $ n $.
\end{proposition}

\begin{proof}
	We use induction. We first verify the base case $ n = 0 $, i.e., we prove $ P(0) $. (Insert proof of $ P(0) $ here). Now suppose inductively that $ n $ is a natural number, and $ P(n) $ has been proven. We now prove $ P(n\inc) $. (Insert proof of $ P(n\inc) $, assuming $ P(n) $ is true, here). This closes the induction, and this $ P(n) $ is true for all natural numbers $ n $.
\end{proof}

\section{Addition}%
\label{sec:Addition}

\begin{definition}[Addition of natural numbers]
    Let $ n $ and $ m $ be natural numbers. We define $ 0 + m \coloneqq m$ and $ (n\inc) + m \coloneqq (n+m)\inc $.
\end{definition}

\begin{proposition}
    Sum of two natural numbers is a natural number.
\end{proposition}

\begin{proof}
	Let $ n \text{ and } m $ be natural numbers. We shall induct on $ n $, with the base case $ n = 0 $. Since, for the base case, either $ m = 0 \text{ or } m \neq 0 $, we can show, be the definition of addition, that $ 0 + 0 = 0 \text{ and } m + 0 = 0 $. Thus the base case holds. 
	Let the hypothesis be true for a non-zero $ n \text{ s.t. } n+m $ is a natural number. We wish to prove that $ (n\inc)+m $ is also a natural number. But note that by definition,
	\begin{equation*}
		(n\inc)+m = (n+m)\inc
	\end{equation*}
	Since by induction hypothesis, $ (n+m) $ is a natural number, by Axiom 2, $ (n+m)\inc $ is also guaranteed to be a natural number. This closes the induction. 
\end{proof}

\begin{lemma}
   For any natural number $ n, n+0 = n $. 
\end{lemma}

\begin{proof}
	Induction over $ n $. Base case: $ 0 + 0 = 0 $. Assume $ n + 0 = n $. Then $ (n\inc)+0 = (n+0)\inc = n\inc $. 
\end{proof}

\begin{lemma}
    For any natural numbers $ n \text{ and } m, n+(m\inc) = (n+m)\inc$. 
\end{lemma}

\begin{proof}
	Induct over $ n $, keeping $ m $ fixed. Base case: $ 0+(m\inc) = (0+m)\inc = m\inc$ by definition. Assume $ n + (m\inc) = (n+m)\inc $. Then $ (n\inc) + (m\inc) = (n+(m\inc))\inc = ((
	n+m)\inc)\inc$. This was the LHS. For the RHS, $ ((n\inc) + m)\inc = ((n+m)\inc)\inc $. Since both are equal, it closes our induction. 
\end{proof}

\begin{proposition}[Addition is commutative]
	For any natural numbers $ n \text{ and } m $, $ n+m = m+n $.
\end{proposition}

\begin{proof}
    Induction on $ n $, keeping $ m $ fixed. Base case: $ 0+m = m = m+0 $, by definition and Lemma 3.1. Assume $ n+m = m+n $. Then $ (n\inc) + m = (n+m)\inc $, and $ m+(n\inc) = (m+n)\inc $ by Lemma 3.2. Since we assume $n+m = m+n$, we can say that $ (n+m)\inc = (m+n)\inc $ (by Axiom 4?). This closes the induction.
\end{proof}

Similarly we have, 

\begin{proposition}[Cancellation law]
    Let $ a, b, c $ be natural numbers such that $ a+b=a+c $. Then we have $ b=c $.
\end{proposition}

\section{Multiplication}%
\label{sec:Multiplication}

\begin{definition}[Multiplication of natural numbers]
	Let $ m $ be a natural number. Define $ 0 \times m \coloneqq 0 $. Now suppose inductively we have shown how to multiply $ n $ to $ m $. Then we can multiply $ n\inc $ to $ m $ by defining $ (n\inc) \times m = n \times m + m $.
\end{definition}

\begin{notation}
	We will now ditch the $ n \times m $ notation. 
\end{notation}

\begin{proposition}
    Multiplication is commutative, associative, distributive and preserves order. Cancellation law is also valid just like in addition
\end{proposition}

\begin{proposition}[Euclidean Algorithm]
    Let $ n $ be a natural number and let $ q $ be a positive number. Then there exist natural numbers $ m, r $ such that $ n = mq + r $ where $ 0 \leq r < q $.
\end{proposition}

Note that uniqueness is not mentioned. 

\end{document}
