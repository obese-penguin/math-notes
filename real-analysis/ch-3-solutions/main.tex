\documentclass[12pt]{article}
\usepackage{graphicx} % Required for inserting images
\usepackage[a4paper, total={6in, 10in}]{geometry}
\usepackage{datetime, xcolor}
\usepackage{amsthm, thmtools, amsfonts, mathtools}

\newdateformat{monthyeardate}{\monthname[\THEMONTH] \THEYEAR}

\newtheorem{theorem}{Theorem}[section]
\newtheorem{corollary}{Corollary}[theorem]
\newtheorem{lemma}[theorem]{Lemma}
\newtheorem{axiom}{Axiom}
\newtheorem{proposition}{Proposition}[section]
\newtheorem{definition}{Definition}[section]
\theoremstyle{remark}
\newtheorem*{remark}{Remark}

\newcommand{\exercise}[1]{\noindent {\bf Exercise #1.}}

\title{Set Theory - Solutions}
\author{Riddhiman}
\date{\monthyeardate\today}

\begin{document}

\maketitle

\section{Preface}%
\label{sec:Preface}

Nothing to put here for now.

\section{Fundamentals}%
\label{sec:Fundamentals}

\exercise{3.1.1} Show that the definition of equality in Definition 3.1.4 (Definition 2.2 in notes) is reflexive, symmetric, and transitive.

\begin{proof}
    Let $ A $, $ B $, $ C $ be sets. To prove reflexivity we have to prove that $ \forall x (x \in A \iff x \in A) $. Since this statement is obviously true we have $ A = A $.
    \par 
    To prove symmetry, consider $ A $ and $ B $ such that $ A = B $. By definition we have $ \forall x (x \in A \iff x \in B) $. We have to show that $ \forall y (y \in B \iff y \in A) $. However, note that the result follows since the two statements are logically equivalent. Therefore, $ B = A $.
    \par
    To prove transitivity, note that for some $ x \in A $ we have $ x \in B $ since $ A = B $ and because $ B = C $ we have $ x \in C $. Similarly for some $ y \in C $ we have $ y \in B $ and thus $ y \in A $. Thus by definition we get $ A = C $.
\end{proof}


\exercise{3.1.2} Using only Definition 3.1.4 and Axiom 3.1, Axiom 3.2 and Axiom 3.3, prove that $ \emptyset $, $ \{ \emptyset \} $ , $ \{ \{ \emptyset \} \} $ and $ \{ \emptyset , \{ \emptyset \} \} $ are all distinct. 

\begin{proof}
	If $ \emptyset $ is equal to the other sets (say $ \mathcal{S} $) we have to show that $ \forall x (x \in \emptyset \iff x \in \mathcal{S}) $. However, note that $ \emptyset $ is empty while the others are non-empty. Hence, $ \emptyset $ cannot be equalt to the other sets. \par
	For the rest of the sets, Axiom 1 and Axiom 2 guarantee us to compare them with Definition 3.1.4. The sets $ \{ \emptyset \} $ (say $ A $) and $ \{ \{ \emptyset \} \} $ (say $ B $) are not equal since $ \emptyset \notin B $ and $ \{ \emptyset \} \notin A $. Hence, $ \{ \emptyset \} \neq \{ \{ \emptyset \} \}$. The other pairs can be proven similarly.  
\end{proof}


\end{document}

